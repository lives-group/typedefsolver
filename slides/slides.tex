\documentclass{beamer}
\usepackage[utf8]{inputenc}
\usepackage[T1]{fontenc}
\usepackage[brazil]{babel}
\usepackage{amsmath,amsfonts,amssymb,amsthm}
\usepackage{proof}
\usepackage{color}
 
\usetheme{Luebeck}

\newcommand{\constr}[1]{\ensuremath{\langle\langle\,#1\,\rangle\rangle}}
 
\title{Inferência de typedef's para C}
\author[Rodrigo Ribeiro]{Rodrigo Ribeiro}
\date{\today}

\begin{document}
   \begin{frame}
        \titlepage
   \end{frame}
   \begin{frame}{Sumário}
        \tableofcontents
   \end{frame}
   \section{Visão geral do algoritmo}
   \begin{frame}{Visão Geral do Algoritmo}
        \begin{itemize}
            \item Algoritmo baseado em inferência de tipos para ML
            \item Dividido em duas etapas
              \begin{itemize}
                  \item Gerar restrições
                  \item Solucionar restrições
             \end{itemize}
           \item Solução das restrições produz, como efeito colateral, definições de
             typedefs.
        \end{itemize}
    \end{frame}
   \section{Sintaxe da linguagem núcleo}
   \begin{frame}{Metavariáveis}
     \begin{table}
     \begin{tabular}{|l|l|}
       \hline
       Símbolo & Significado \\ \hline
		$l$ & {literal} \\
		$x$ & {variável}\\
		$f$  & função \\
		$\tau$ & {tipo}\\
		$\circ$ & {operador binário qualquer}\\
                $\rho : l \to \tau$ & {função para atribuir tipos a literais}\\
		\hline
		\end{tabular}
           \centering
       \end{table}
   \end{frame}
   \begin{frame}{Expressões}
\[
\begin{array}{lcll}
	e & ::= & l & \{\text{literal}\} \\
	& \mid & x & \{\text{variável}\} \\
	& \mid & e . x & \{\text{acesso a campo}\}\\
	& \mid & e\,[e_1] & \{\text{accesso arranjo}\} \\
	& \mid & (\tau)\,e & \{\text{casting}\} \\
	&\mid & e\circ e' & \{\text{bin op}\} \\
	& \mid & \&\,e & \{\text{address}\} \\
	& \mid & \star\,e & \{\text{deref. ponteiro}\}\\
	& \mid & e \to x & \{\text{ref. campo ponteiro}\}\\
	& \mid & f(e_i)^{i=0..n} & \{\text{chamada função}\}\\
\end{array} \]
   \end{frame}
   \begin{frame}{Comandos}
     \[
\begin{array}{lcll}
	s & ::=   & \tau\:x\:=\:e & \{\text{var. def.}\} \\
	& \mid & x\:=\:e & \{\text{atrib. var.}\} \\
	& \mid & \star\,x\:=\:e & \{\text{atrib. pont.}\} \\
	& \mid & x[e]\:=\:e &\{\text{atrib. arranjo}\} \\
\end{array} \]
   \end{frame}
   \begin{frame}{Declarações}
     \begin{itemize}
         \item typedefs são apenas um par formado por um tipo e um nome.
         \item Sintaxe de funções
         \begin{itemize}
             \item Formada por: tipo de retorno, nome, parâmetros, comandos
             \item Último comando deve ser um retorno (restrição para facilitar
               geração de restrições).
         \end{itemize}
     \end{itemize}
   \end{frame}
   \begin{frame}{Tipos}
     \[
\begin{array}{lcll}
	\tau & ::= & \textbf{B} & \{\text{tipos básicos: void, int,
	etc.}\}\\
	& \mid & \star\, \tau & \{\text{ponteiros}\} \\
	& \mid & \{x_i : \tau_i\}^{i=1..n} & \{\text{registros}\} \\
	& \mid & \tau[n] & \{\text{arranjo}\}\\
        & \mid & \tau^{0..n} \to\tau & \{\text{tipos de funções}\}\\
\end{array}
     \]
   \end{frame}
   \section{Sintaxe de restrições}
   \begin{frame}{Sintaxe de restrições}
       \begin{itemize}
           \item Existência de declaração: $def\:x\,:\,\tau$.
           \item Igualdade de tipos: $\tau \equiv \tau'$.
	   \item Existencia de campos: $has(\tau,x:\tau')$.
	   \item Arranjos: $\tau \equiv\tau'[]$.
	   \item Pointeiros: $\tau \equiv \star\tau'$.
       \end{itemize}
   \end{frame}
   \section{Geração de restrições}
   \begin{frame}{Geração de restrições para expressões}
       \[\small{
            \begin{array}{lcl}
                \constr{l : \tau} & = & \rho(l) \equiv \tau\\
                \constr{x : \tau} & = & x \equiv \tau \\
                \constr{\circ : \tau} & = & \circ \equiv \tau \\
                \constr{f : \tau^{0..n}\to \tau'} & = & f \equiv
                                                        \tau^{0..n}\to
                                                        \tau' \\
                \constr{e.x : \tau} & = & \exists
                                          \alpha_1\,\alpha_2. \constr{e
                                          : \alpha_1} \land \constr{x
                                          : \alpha_2} \land \\
                          & & has(\alpha_1,x:\alpha_2) \land \tau
                              \equiv \alpha_2\\
                \constr{e[e_1] : \tau} & = & \exists
                                         \alpha_1\alpha_2\alpha_3. \constr{e:\alpha_1}
                                             \land\constr{e_1:\alpha_2}\land\\
                         & & \alpha_1\equiv\alpha_3[]\land \tau \equiv
                             \alpha_3\\
                \constr{(\tau')\,e:\tau} & = & \constr{e : \tau} \land
                                               \tau \equiv \tau'\\
                \constr{f(e^{i=0..n}) : \tau} & = & \exists
                                                    \alpha^{i=0..n}. \bigwedge_{i=0..n}\constr{e_i:\alpha_i}
                                                    \land \constr{f :
                                                    \alpha^{i=0..n}
                                                    \to \tau} 
            \end{array}}
       \]
   \end{frame}
   \begin{frame}{Geração de restrições para expressões}
       \[
            \small{
                 \begin{array}{lcl}
                     \constr{e \circ e' : \tau} & = & \exists
                                                      \alpha_1\alpha_2.\constr{e
                                                      :\alpha_1} \land
                                                      \constr{e' :
                                                      \alpha_2} \land
                                                      \constr{\circ :
                                                      \alpha_1\to\alpha_2\to\tau}\\
                     \constr{\&\,e : \tau} & = & \exists \alpha
                                                 \alpha'. \constr{e :
                                                 \alpha} \land
                                                 \alpha \equiv \star \alpha'
                                                 \land \tau = int\\
                     \constr{\star\,e : \tau} & = & \exists
                                                    \alpha. \constr{e
                                                    : \alpha} \land
                                                    \alpha = \star \tau\\
                     \constr{e\to x : \tau} & = & \exists
                                                  \alpha_1\alpha_2\alpha_3. \constr{e
                                                  :\alpha_1} \land
                                                  \constr{x :
                                                  \alpha_3} \land
                                                  \alpha_1 = \star\alpha_2
                                                  \land\\
                          & & has(\alpha,x : \alpha_3) \land \tau
                              \equiv \alpha_3\\
                 \end{array}}
       \]
   \end{frame}
   \begin{frame}{Geração de restrições para comandos}
       \[
          \small{
           \begin{array}{lcl}
               \constr{\emptyset} & = & true \\
               \constr{\tau\:x := e; S} & = & \exists \alpha. \constr{e : \alpha} \land
                                              def\:\tau\: in\:
                                              def\,x:\tau\,in\,\constr{S}
             \land \tau \equiv \alpha\\
               \constr{x := e ; S} & = & \exists \alpha. \constr{x : \alpha}\land\constr{e : \alpha} \land
                                                      \constr{S}\\ 
               \constr{\star\,x := e ; S} & = & \exists \alpha \alpha'. \constr{x : \alpha'}\land\constr{e : \alpha} \land
                                                      \alpha' \equiv
                                                \star \alpha)\land\constr{S}\\ 
               \constr{x[e'] := e ; S} & = &
                                             \exists\alpha_1\alpha_2\alpha_3.\constr{e'
                                             : \alpha_1}\land \alpha_1
                                             = int \land \constr{e :
                                             \alpha_2} \land \\ 
                       & & \constr{x : \alpha_3} \land
                           \alpha_3 \equiv \alpha_2[] \\
           \end{array}}
       \]
   \end{frame}
   \begin{frame}{Geração de restrições para funções e definições}
       \[
           \small{
                \begin{array}{lcl}
              \constr{\tau\:f(\tau_i\:x_i)^{i=0..n}\:\:S\:\:e} & = &
                                                                     \exists
                                                                     \alpha. def\:f:\tau^{i=0..n}\to\tau\:\:in\:\:def\:\:x_i:\tau_i^{i=0..n}\:in\\
                      & &\constr{S}\land\constr{e : \alpha} \land
                          \alpha \equiv \tau\\
              \constr{typedef\:\:\tau\: x} & = & def\:x\,=\,\tau\\
                \end{array}
           }
       \]
   \end{frame}
   \section{Solucionar restrições }
   \begin{frame}{Solucionando restrições --- Visão geral}
       \begin{itemize}
           \item Restrições são basicamente restrições de igualdade e,
             portanto, podem ser resolvidas por unificação.
             \begin{itemize}
                  \item Quantificadores existenciais correspondem a
                    variáveis ``fresh'', i.e., novas variáveis.
                  \item Restrições de definição substituem o nome de
                    um identificador pelo tipo associado a este.
                  \item Restrições de igualdade: resolvidas por unificação.
             \end{itemize}
       \end{itemize}
   \end{frame}
   \begin{frame}{Solucionando restrições --- Visão geral}
      \begin{itemize}
           \item Restrições de campos
           \begin{itemize}
                \item Consideradas após a solução de todas as
                  restrições de igualdade.
                \item Substituições geradas por unificação devem ser
                  aplicadas a restrições ainda não processadas.
           \end{itemize}       
      \end{itemize}
   \end{frame}
   \begin{frame}{Solucionando restrições --- Unificação}
   \end{frame}
   \begin{frame}{Limitações conhecidas}
     \begin{itemize}
          \item Dimensão de arranjos não são tratadas.
          \item Ponteiros para funções.
          \item Algo que ainda não percebi? :)
     \end{itemize}
   \end{frame}
\end{document}